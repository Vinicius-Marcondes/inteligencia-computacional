% =======================================================================
% =                                                                     =
% = ABNTEX - UTP                                                        =
% =                                                                     =
% =======================================================================
% -----------------------------------------------------------------------
% Author: Chaua Queirolo
% Data:   01/07/2017
% -----------------------------------------------------------------------
\documentclass[12pt,oneside,a4paper,chapter=TITLE,section=TITLE,sumario
=tradicional]{abntex2}

% Regras da abnt
\usepackage{packages/abnt-UTP}
\usepackage{lipsum}

% =======================================================================
% =                                                                     =
% = DADOS DO TRABALHO                                                   =
% =                                                                     =
% =======================================================================

% Informações de dados para CAPA e FOLHA DE ROSTO
\titulo{Como o Estudo de Ondas Gravitacionais Contribui para a Medição da Expansão do Universo}

\autor{Vinícius Marcondes Vieira}

\orientador{Prof. Chauã Queirolo}

\preambulo{Trabalho de Conclusão de Curso apresentado ao curso de Bacharelado 
em Ciência da Computação da Faculdade de Ciências Exatas e de Tecnologia da 
Universidade Tuiuti do Paraná, como requisito à obtenção ao grau de Bacharel.}

\instituicao{Universidade Tuiuti do Paraná}
\local{Curitiba}
\data{2023}

% =======================================================================
% =                                                                     =
% = DOCUMENTO                                                           =
% =                                                                     =
% =======================================================================
\begin{document}

% -----------------------------------------------------------------------
% -                                                                     -
% - ELEMENTOS PRÉ-TEXTUAIS                                              -
% -                                                                     -
% -----------------------------------------------------------------------

% Capa e folha de rosto
\imprimircapa
\imprimirfolhaderosto

% Sumario
\sumario

% -----------------------------------------------------------------------
% -                                                                     -
% - ELEMENTOS TEXTUAIS                                                  -
% -                                                                     -
% -----------------------------------------------------------------------
% Inicia a numeracao das páginas
\textual

% -----------------------------------------------------------------------
% -----------------------------------------------------------------------
\chapter{Introdução}
\label{cap:introducao}


O estudo de ondas gravitacionais, previstas pela Teoria da Relatividade Geral de Albert Einstein \cite{bertolami2017}, tem ampliado o escopo da astrofísica e da cosmologia. A detecção dessas ondas, que são perturbações no espaço-tempo geradas por eventos astrofísicos extremos, foi realizada pela primeira vez em 2015 pela colaboração LIGO \cite{bertolami2017, aguiar2018}. Essa conquista abriu novas possibilidades para a observação do universo, complementando métodos tradicionais baseados em radiação eletromagnética.

Um dos campos que mais se beneficiou desse avanço é a cosmologia, especialmente na medição da taxa de expansão do universo, descrita pela constante de Hubble. Métodos convencionais para a medição dessa constante, como o estudo de supernovas Tipo Ia e o mapeamento do fundo cósmico de micro-ondas, apresentam limitações \cite{silva2021}. Ondas gravitacionais oferecem uma alternativa, permitindo a medição direta do \emph{redshift} e, consequentemente, uma estimativa mais precisa da constante de Hubble \cite{silva2021}.

Este artigo tem como objetivo explorar a contribuição das ondas gravitacionais na medição da expansão do universo. Serão abordados os princípios fundamentais das ondas gravitacionais, os métodos de detecção e como essas detecções podem ser aplicadas para estimar parâmetros cosmológicos \cite{ramos2019, silva2021}. A análise se baseia em uma revisão da literatura atual e nos mais recentes desenvolvimentos na área.

% -----------------------------------------------------------------------
% -----------------------------------------------------------------------
\chapter{Fundamentação Teórica}
\label{cap:fundamentacao-teorica}

Esta seção abordará os conceitos teóricos que fundamentam o estudo de ondas gravitacionais e sua aplicação na medição da expansão do universo. Serão discutidas as bases da Teoria da Relatividade Geral, a natureza e detecção de ondas gravitacionais, bem como os métodos atuais para estimar a constante de Hubble. O objetivo é fornecer o contexto teórico necessário para compreender as implicações práticas e cosmológicas desses fenômenos.

% -----------------------------------------------------------------------
% -----------------------------------------------------------------------
\chapter{Revisão da Literatura}
\label{cap:revisao-literatura}

Esta seção apresenta uma revisão de estudos e pesquisas anteriores que exploram o papel das ondas gravitacionais na astrofísica e na cosmologia. O foco será em trabalhos que abordam a medição da constante de Hubble e outros parâmetros cosmológicos através do uso de ondas gravitacionais, fornecendo um panorama do estado atual do campo e identificando lacunas que o presente estudo busca preencher.

A tabela \ref{table:related_work} lista os principais trabalhos estudados para a realização do presente artigo.

\begin{table}[h]
\centering
\caption{Trabalhos Relacionados}
\label{table:related_work}
\begin{tabular}{|c|c|c|}
\hline
Autor(es) & Ano & Foco do Estudo \\
\hline
Bertolami \& Gomes & 2017 & Fundamentos de Ondas Gravitacionais \\
\hline
Aguiar & 2018 & Detecção e Fontes de Ondas Gravitacionais \\
\hline
Ramos \& Maluf & 2019 & Radiação Emitida por Pulsares Binários \\
\hline
Silva & 2021 & Aplicações em Cosmologia \\
\hline
\end{tabular}
\fonte{O próprio autor, 2023}
\end{table}


% -----------------------------------------------------------------------
% -----------------------------------------------------------------------
\chapter{Metodologia}
\label{cap:metodologia}

Esta seção descreve a metodologia adotada para a realização deste estudo, que se baseia em uma pesquisa bibliográfica abrangente. Foram selecionados artigos científicos, relatórios e outros documentos acadêmicos que abordam o papel das ondas gravitacionais na astrofísica e na cosmologia, com ênfase na medição da constante de Hubble. O objetivo é sintetizar os métodos e descobertas mais relevantes da literatura para fornecer uma visão consolidada do estado atual do campo.

A \autoref{fig:interferometro} ilustra um Interferômetro de Michelson, instrumento fundamental na detecção de ondas gravitacionais. Este dispositivo é empregado para quantificar minúsculas flutuações no espaço-tempo.
\begin{figure}[h]
    \legenda[fig:interferometro]{Interferómetro de Michelso}
    \fig{scale=0.6}{imagens/deteccao_de_ondas_gravitacionais.png}
    \fonte{\cite{bertolami2017}}
\end{figure}

% -----------------------------------------------------------------------
% -----------------------------------------------------------------------
\chapter{Resultados Experimentais}
\label{cap:resultados}

Esta seção apresenta os resultados obtidos a partir da pesquisa bibliográfica realizada, focando em como as ondas gravitacionais têm sido utilizadas para medir a expansão do universo. Serão discutidas as principais descobertas, métodos e implicações desses estudos, permitindo uma avaliação crítica da eficácia e precisão das técnicas baseadas em ondas gravitacionais para estimar parâmetros cosmológicos como a constante de Hubble.

A tabela \ref{table:results} resume os principais resultados obtidos a partir da pesquisa bibliográfica realizada.

\begin{table}[h]
\centering
\caption{Resultados Baseados nos Artigos Lidos}
\label{table:results}
\begin{tabularx}{\textwidth}{|X|X|X|}
\hline
Autor(es) & Resultado Principal & Implicações \\
\hline
Bertolami \& Gomes & Detecção direta de ondas gravitacionais & Novas possibilidades em astrofísica \\
\hline
Aguiar & Discussão sobre métodos de detecção & Avanços na precisão da detecção \\
\hline
Ramos \& Maluf & Teoria aplicada a pulsares binários & Compreensão da radiação gravitacional \\
\hline
Silva & Método para medir a constante de Hubble & Precisão na medição da expansão do universo \\
\hline
\end{tabularx}
\fonte{O próprio autor, 2023}
\end{table}




\chapter{Conclusão}

\bibliography{referencias}
\nocite{smith2022}
\nocite{johnson2021}
\nocite{brown2020}


\end{document}
